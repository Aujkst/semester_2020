\documentclass[]{article}
\usepackage{lmodern}
\usepackage{amssymb,amsmath}
\usepackage{ifxetex,ifluatex}
\usepackage{fixltx2e} % provides \textsubscript
\ifnum 0\ifxetex 1\fi\ifluatex 1\fi=0 % if pdftex
  \usepackage[T1]{fontenc}
  \usepackage[utf8]{inputenc}
\else % if luatex or xelatex
  \ifxetex
    \usepackage{mathspec}
  \else
    \usepackage{fontspec}
  \fi
  \defaultfontfeatures{Ligatures=TeX,Scale=MatchLowercase}
\fi
% use upquote if available, for straight quotes in verbatim environments
\IfFileExists{upquote.sty}{\usepackage{upquote}}{}
% use microtype if available
\IfFileExists{microtype.sty}{%
\usepackage{microtype}
\UseMicrotypeSet[protrusion]{basicmath} % disable protrusion for tt fonts
}{}
\usepackage[margin=1in]{geometry}
\usepackage{hyperref}
\hypersetup{unicode=true,
            pdftitle={Time Series Analysis},
            pdfauthor={shake\_it},
            pdfborder={0 0 0},
            breaklinks=true}
\urlstyle{same}  % don't use monospace font for urls
\usepackage{color}
\usepackage{fancyvrb}
\newcommand{\VerbBar}{|}
\newcommand{\VERB}{\Verb[commandchars=\\\{\}]}
\DefineVerbatimEnvironment{Highlighting}{Verbatim}{commandchars=\\\{\}}
% Add ',fontsize=\small' for more characters per line
\usepackage{framed}
\definecolor{shadecolor}{RGB}{248,248,248}
\newenvironment{Shaded}{\begin{snugshade}}{\end{snugshade}}
\newcommand{\AlertTok}[1]{\textcolor[rgb]{0.94,0.16,0.16}{#1}}
\newcommand{\AnnotationTok}[1]{\textcolor[rgb]{0.56,0.35,0.01}{\textbf{\textit{#1}}}}
\newcommand{\AttributeTok}[1]{\textcolor[rgb]{0.77,0.63,0.00}{#1}}
\newcommand{\BaseNTok}[1]{\textcolor[rgb]{0.00,0.00,0.81}{#1}}
\newcommand{\BuiltInTok}[1]{#1}
\newcommand{\CharTok}[1]{\textcolor[rgb]{0.31,0.60,0.02}{#1}}
\newcommand{\CommentTok}[1]{\textcolor[rgb]{0.56,0.35,0.01}{\textit{#1}}}
\newcommand{\CommentVarTok}[1]{\textcolor[rgb]{0.56,0.35,0.01}{\textbf{\textit{#1}}}}
\newcommand{\ConstantTok}[1]{\textcolor[rgb]{0.00,0.00,0.00}{#1}}
\newcommand{\ControlFlowTok}[1]{\textcolor[rgb]{0.13,0.29,0.53}{\textbf{#1}}}
\newcommand{\DataTypeTok}[1]{\textcolor[rgb]{0.13,0.29,0.53}{#1}}
\newcommand{\DecValTok}[1]{\textcolor[rgb]{0.00,0.00,0.81}{#1}}
\newcommand{\DocumentationTok}[1]{\textcolor[rgb]{0.56,0.35,0.01}{\textbf{\textit{#1}}}}
\newcommand{\ErrorTok}[1]{\textcolor[rgb]{0.64,0.00,0.00}{\textbf{#1}}}
\newcommand{\ExtensionTok}[1]{#1}
\newcommand{\FloatTok}[1]{\textcolor[rgb]{0.00,0.00,0.81}{#1}}
\newcommand{\FunctionTok}[1]{\textcolor[rgb]{0.00,0.00,0.00}{#1}}
\newcommand{\ImportTok}[1]{#1}
\newcommand{\InformationTok}[1]{\textcolor[rgb]{0.56,0.35,0.01}{\textbf{\textit{#1}}}}
\newcommand{\KeywordTok}[1]{\textcolor[rgb]{0.13,0.29,0.53}{\textbf{#1}}}
\newcommand{\NormalTok}[1]{#1}
\newcommand{\OperatorTok}[1]{\textcolor[rgb]{0.81,0.36,0.00}{\textbf{#1}}}
\newcommand{\OtherTok}[1]{\textcolor[rgb]{0.56,0.35,0.01}{#1}}
\newcommand{\PreprocessorTok}[1]{\textcolor[rgb]{0.56,0.35,0.01}{\textit{#1}}}
\newcommand{\RegionMarkerTok}[1]{#1}
\newcommand{\SpecialCharTok}[1]{\textcolor[rgb]{0.00,0.00,0.00}{#1}}
\newcommand{\SpecialStringTok}[1]{\textcolor[rgb]{0.31,0.60,0.02}{#1}}
\newcommand{\StringTok}[1]{\textcolor[rgb]{0.31,0.60,0.02}{#1}}
\newcommand{\VariableTok}[1]{\textcolor[rgb]{0.00,0.00,0.00}{#1}}
\newcommand{\VerbatimStringTok}[1]{\textcolor[rgb]{0.31,0.60,0.02}{#1}}
\newcommand{\WarningTok}[1]{\textcolor[rgb]{0.56,0.35,0.01}{\textbf{\textit{#1}}}}
\usepackage{graphicx,grffile}
\makeatletter
\def\maxwidth{\ifdim\Gin@nat@width>\linewidth\linewidth\else\Gin@nat@width\fi}
\def\maxheight{\ifdim\Gin@nat@height>\textheight\textheight\else\Gin@nat@height\fi}
\makeatother
% Scale images if necessary, so that they will not overflow the page
% margins by default, and it is still possible to overwrite the defaults
% using explicit options in \includegraphics[width, height, ...]{}
\setkeys{Gin}{width=\maxwidth,height=\maxheight,keepaspectratio}
\IfFileExists{parskip.sty}{%
\usepackage{parskip}
}{% else
\setlength{\parindent}{0pt}
\setlength{\parskip}{6pt plus 2pt minus 1pt}
}
\setlength{\emergencystretch}{3em}  % prevent overfull lines
\providecommand{\tightlist}{%
  \setlength{\itemsep}{0pt}\setlength{\parskip}{0pt}}
\setcounter{secnumdepth}{0}
% Redefines (sub)paragraphs to behave more like sections
\ifx\paragraph\undefined\else
\let\oldparagraph\paragraph
\renewcommand{\paragraph}[1]{\oldparagraph{#1}\mbox{}}
\fi
\ifx\subparagraph\undefined\else
\let\oldsubparagraph\subparagraph
\renewcommand{\subparagraph}[1]{\oldsubparagraph{#1}\mbox{}}
\fi

%%% Use protect on footnotes to avoid problems with footnotes in titles
\let\rmarkdownfootnote\footnote%
\def\footnote{\protect\rmarkdownfootnote}

%%% Change title format to be more compact
\usepackage{titling}

% Create subtitle command for use in maketitle
\providecommand{\subtitle}[1]{
  \posttitle{
    \begin{center}\large#1\end{center}
    }
}

\setlength{\droptitle}{-2em}

  \title{Time Series Analysis}
    \pretitle{\vspace{\droptitle}\centering\huge}
  \posttitle{\par}
    \author{shake\_it}
    \preauthor{\centering\large\emph}
  \postauthor{\par}
    \date{}
    \predate{}\postdate{}
  
%\documentclass{article} 
%\usepackage{ctex} 
\usepackage{latexsym,bm}
\usepackage[BoldFont,SlantFont,CJKchecksingle]{xeCJK}
\setCJKmainfont[BoldFont=SimSun]{Microsoft YaHei} %雅黑
\setCJKmonofont{SimSun}% 设置缺省中文字体
\parindent 2em   %段首缩进

\begin{document}
\maketitle

\hypertarget{ux7b2cux4e00ux7ae0-ux65f6ux95f4ux5e8fux5217ux5206ux6790ux7b80ux4ecb}{%
\section{第一章
时间序列分析简介}\label{ux7b2cux4e00ux7ae0-ux65f6ux95f4ux5e8fux5217ux5206ux6790ux7b80ux4ecb}}

\hypertarget{ux65b9ux6cd5}{%
\paragraph{方法}\label{ux65b9ux6cd5}}

\begin{itemize}
\tightlist
\item
  \textbf{描述性时序分析}
\item
  \textbf{统计时序分析}(频域分析方法 + 时域分析方法)
\end{itemize}

\hypertarget{ux751fux6210ux6570ux636e}{%
\paragraph{生成数据}\label{ux751fux6210ux6570ux636e}}

从2005年1月开始的月度数据。\texttt{start}指定起始读入时间,\texttt{frequency}指定序列每年读入的数据频率。

\begin{Shaded}
\begin{Highlighting}[]
\NormalTok{price <-}\StringTok{ }\KeywordTok{c}\NormalTok{(}\DecValTok{101}\NormalTok{, }\DecValTok{82}\NormalTok{, }\DecValTok{66}\NormalTok{, }\DecValTok{35}\NormalTok{, }\DecValTok{31}\NormalTok{, }\DecValTok{7}\NormalTok{)}

\NormalTok{price <-}\StringTok{ }\KeywordTok{ts}\NormalTok{(price, }\DataTypeTok{start =} \KeywordTok{c}\NormalTok{(}\DecValTok{2005}\NormalTok{, }\DecValTok{1}\NormalTok{), }\DataTypeTok{frequency =} \DecValTok{12}\NormalTok{)}
\end{Highlighting}
\end{Shaded}

\begin{center}\rule{0.5\linewidth}{\linethickness}\end{center}

\hypertarget{ux4f8b1-1}{%
\paragraph{例1-1}\label{ux4f8b1-1}}

读入1884-1939年英格兰和威尔士小麦平均亩产量数据\texttt{file1.csv}

\begin{Shaded}
\begin{Highlighting}[]
\NormalTok{x <-}\StringTok{ }\KeywordTok{read.csv}\NormalTok{(}\StringTok{"D:/Documents/UIBE/6/TimeSeries/file1.csv"}\NormalTok{)}
\KeywordTok{head}\NormalTok{(x)}
\end{Highlighting}
\end{Shaded}

\begin{verbatim}
##   year yield
## 1 1884  15.2
## 2 1885  16.9
## 3 1886  15.3
## 4 1887  14.9
## 5 1888  15.7
## 6 1889  15.1
\end{verbatim}

截取1925年之后的数据\texttt{subset}

\begin{Shaded}
\begin{Highlighting}[]
\NormalTok{z <-}\StringTok{ }\KeywordTok{subset}\NormalTok{(x, year }\OperatorTok{>}\StringTok{ }\DecValTok{1925}\NormalTok{, }\DataTypeTok{select =}\NormalTok{ yield)}
\KeywordTok{head}\NormalTok{(z)}
\end{Highlighting}
\end{Shaded}

\begin{verbatim}
##    yield
## 43  16.0
## 44  16.4
## 45  17.2
## 46  17.8
## 47  14.4
## 48  15.0
\end{verbatim}

对\texttt{yield}序列进行对数变换,并将对数序列和原序列值导出,保存为数据文件\texttt{yield.csv}。

\begin{Shaded}
\begin{Highlighting}[]
\NormalTok{ln_yield <-}\StringTok{ }\KeywordTok{log}\NormalTok{(x}\OperatorTok{$}\NormalTok{yield)}
\NormalTok{x_new <-}\StringTok{ }\KeywordTok{data.frame}\NormalTok{(x,ln_yield) }\CommentTok{#新数据框}
\KeywordTok{write.csv}\NormalTok{(x_new, }\DataTypeTok{file =} \StringTok{"D:/Documents/UIBE/6/TimeSeries/yield.csv"}\NormalTok{, }\DataTypeTok{row.names =}\NormalTok{ F)}
\end{Highlighting}
\end{Shaded}

\begin{center}\rule{0.5\linewidth}{\linethickness}\end{center}

\hypertarget{ux7f3aux5931ux503cux63d2ux503c}{%
\paragraph{缺失值插值}\label{ux7f3aux5931ux503cux63d2ux503c}}

R中缺失值用\texttt{NA}表示。常用的插值方法:\textbf{线性插值}和\textbf{样条插值}

\begin{Shaded}
\begin{Highlighting}[]
\KeywordTok{library}\NormalTok{(zoo)}
\NormalTok{a <-}\StringTok{ }\DecValTok{1}\OperatorTok{:}\DecValTok{7}
\NormalTok{a[}\DecValTok{4}\NormalTok{] <-}\StringTok{ }\OtherTok{NA}
\KeywordTok{cat}\NormalTok{(}\StringTok{"a: "}\NormalTok{, a)}
\end{Highlighting}
\end{Shaded}

\begin{verbatim}
## a:  1 2 3 NA 5 6 7
\end{verbatim}

\begin{Shaded}
\begin{Highlighting}[]
\NormalTok{y1 <-}\StringTok{ }\KeywordTok{na.approx}\NormalTok{(a)}
\NormalTok{y2<-}\KeywordTok{na.spline}\NormalTok{(a)}
\KeywordTok{cat}\NormalTok{(}\StringTok{" "}\NormalTok{, y1, }\StringTok{"}\CharTok{\textbackslash{}n}\StringTok{ "}\NormalTok{, y2)}
\end{Highlighting}
\end{Shaded}

\begin{verbatim}
##   1 2 3 4 5 6 7 
##   1 2 3 4 5 6 7
\end{verbatim}

\begin{center}\rule{0.5\linewidth}{\linethickness}\end{center}

\hypertarget{ux7b2cux4e8cux7ae0-ux65f6ux95f4ux5e8fux5217ux6570ux636eux7684ux9884ux5904ux7406}{%
\section{第二章
时间序列数据的预处理}\label{ux7b2cux4e8cux7ae0-ux65f6ux95f4ux5e8fux5217ux6570ux636eux7684ux9884ux5904ux7406}}

\hypertarget{ux5e73ux7a33ux6027ux68c0ux9a8c}{%
\paragraph{平稳性检验}\label{ux5e73ux7a33ux6027ux68c0ux9a8c}}

\hypertarget{ux7edfux8ba1ux6027ux8d28}{%
\paragraph{统计性质}\label{ux7edfux8ba1ux6027ux8d28}}

平稳时间序列\textbf{自协方差}函数和\textbf{自相关}系数只依赖于时间的平移长度而与时间的起止点无关

\begin{itemize}
\tightlist
\item
  \textbf{延迟\(k\)自协方差函数} \[
  \gamma(k) = \gamma(t,t+k),\quad\forall\,k\in\mathbb{N}
  \]
\item
  \textbf{延迟\(k\)自相关系数} \[
  \rho_k = \frac{\gamma(t,t+k)}{\sqrt{DX_t\cdot DX_{t+1}}} = \frac{\gamma(k)}{\gamma(0)}
  \]
\item
  \textbf{估计}均值函数 \[
  \hat{\mu} = \bar{x} = \frac{\sum\limits_{i=1}^{n}x_i}{n}
  \]
\item
  \textbf{估计}延迟\(k\)自相关系数 \[
  \hat{\rho_k} = \frac{\sum\limits_{i=1}^{n-k}(x_i-\bar{x})(x_{i+k}-\bar{x})}{\sum\limits_{i=1}^{n}(x_i-\bar{x})^2},\quad\forall\,0<k<n
  \]
\end{itemize}

\hypertarget{ux56fe}{%
\paragraph{图}\label{ux56fe}}

\begin{enumerate}
\def\labelenumi{\arabic{enumi}.}
\tightlist
\item
  时序图
\end{enumerate}

1884-1890年英格兰和威尔士地区小麦平均亩产量

\begin{Shaded}
\begin{Highlighting}[]
\NormalTok{yield <-}\StringTok{ }\KeywordTok{c}\NormalTok{(}\FloatTok{15.2}\NormalTok{,}\FloatTok{16.9}\NormalTok{,}\FloatTok{15.3}\NormalTok{,}\FloatTok{14.9}\NormalTok{,}\FloatTok{15.7}\NormalTok{,}\FloatTok{15.1}\NormalTok{,}\FloatTok{16.7}\NormalTok{)}
\NormalTok{yield <-}\StringTok{ }\KeywordTok{ts}\NormalTok{(yield, }\DataTypeTok{start =} \DecValTok{1884}\NormalTok{)}
\KeywordTok{plot}\NormalTok{(yield,}\DataTypeTok{xlab =} \StringTok{"year"}\NormalTok{, }\DataTypeTok{ylab=}\StringTok{"yield"}\NormalTok{)}
\KeywordTok{abline}\NormalTok{(}\DataTypeTok{v =} \KeywordTok{c}\NormalTok{(}\DecValTok{1885}\NormalTok{,}\DecValTok{1889}\NormalTok{), }
       \DataTypeTok{h =} \KeywordTok{c}\NormalTok{(}\FloatTok{15.5}\NormalTok{,}\FloatTok{16.5}\NormalTok{), }
       \DataTypeTok{lty =} \DecValTok{2}\NormalTok{)}
\end{Highlighting}
\end{Shaded}

\includegraphics{timeseries_files/figure-latex/unnamed-chunk-7-1.pdf}

\texttt{plot}各项参数:

\begin{Shaded}
\begin{Highlighting}[]
\ControlFlowTok{if}\NormalTok{(}\OtherTok{FALSE}\NormalTok{) \{}
\NormalTok{  type =}\StringTok{ "p"}  \CommentTok{# 点}
         \StringTok{"l"}  \CommentTok{# 线}
         \StringTok{"b"}  \CommentTok{# 点连线}
         \StringTok{"o"}  \CommentTok{# 线穿过点}
         \StringTok{"h"}  \CommentTok{# 悬垂线}
         \StringTok{"s"}  \CommentTok{# 阶梯线}
\NormalTok{  pch =}\StringTok{ }\DecValTok{17} \CommentTok{# 点的符号}
\NormalTok{  lty =}\StringTok{ }\DecValTok{2}  \CommentTok{# 连线的类型}
\NormalTok{  lwd =}\StringTok{ }\DecValTok{2}  \CommentTok{# 连线的宽度(默认宽度的2倍)}
\NormalTok{  col =}\StringTok{ }\DecValTok{1}  \CommentTok{# col = "black"}
\NormalTok{  col =}\StringTok{ }\DecValTok{2}  \CommentTok{# col = "red"}
\NormalTok{  col =}\StringTok{ }\DecValTok{3}  \CommentTok{# col = "green"}
\NormalTok{  col =}\StringTok{ }\DecValTok{4}  \CommentTok{# col = "blue"}
\NormalTok{  xlim =}\StringTok{ }\KeywordTok{c}\NormalTok{(}\DecValTok{1886}\NormalTok{,}\DecValTok{1890}\NormalTok{)}
\NormalTok{  ylim =}\StringTok{ }\KeywordTok{c}\NormalTok{(}\DecValTok{15}\NormalTok{,}\DecValTok{16}\NormalTok{) }\CommentTok{# 指定坐标轴范围}
\NormalTok{\}}
\end{Highlighting}
\end{Shaded}

\begin{enumerate}
\def\labelenumi{\arabic{enumi}.}
\setcounter{enumi}{1}
\tightlist
\item
  自相关图
\end{enumerate}

自相关图是一个平面悬垂线图,横坐标表示延迟期数,纵坐标表示自相关系数,悬垂线表示自相关系数的大小。

\begin{Shaded}
\begin{Highlighting}[]
\KeywordTok{acf}\NormalTok{(yield) }\CommentTok{#虚线为自相关系数 2倍标准差位置}
\end{Highlighting}
\end{Shaded}

\includegraphics{timeseries_files/figure-latex/unnamed-chunk-9-1.pdf}

\hypertarget{ux5e73ux7a33ux6027ux7684ux68c0ux9a8cux56feux68c0ux9a8cux65b9ux6cd5}{%
\paragraph{平稳性的检验(图检验方法)}\label{ux5e73ux7a33ux6027ux7684ux68c0ux9a8cux56feux68c0ux9a8cux65b9ux6cd5}}

\begin{itemize}
\item
  \textbf{时序图检验}:始终在一个常数值附近随机波动,而且波动的范围有界、无明显趋势及周期特征。
\item
  \textbf{自相关图检验}:平稳序列通常具有短期相关性。该性质用自相关系数来描述就是随着延迟期数的增加,平稳序列的自相关系数会很快地衰减向零。
\end{itemize}

\begin{center}\rule{0.5\linewidth}{\linethickness}\end{center}

\hypertarget{ux4f8b2.1-ux68c0ux9a8c1964ux5e74-1999ux5e74ux4e2dux56fdux7eb1ux5e74ux4ea7ux91cfux5e8fux5217ux7684ux5e73ux7a33ux6027}{%
\paragraph{例2.1
检验1964年-1999年中国纱年产量序列的平稳性}\label{ux4f8b2.1-ux68c0ux9a8c1964ux5e74-1999ux5e74ux4e2dux56fdux7eb1ux5e74ux4ea7ux91cfux5e8fux5217ux7684ux5e73ux7a33ux6027}}

\begin{Shaded}
\begin{Highlighting}[]
\KeywordTok{library}\NormalTok{(readr)}
\NormalTok{sha <-}\StringTok{ }\KeywordTok{read_csv}\NormalTok{(}\StringTok{"timeseries_data/file4.csv"}\NormalTok{)}
\NormalTok{output <-}\StringTok{ }\KeywordTok{ts}\NormalTok{(sha}\OperatorTok{$}\NormalTok{output, }\DataTypeTok{start =} \DecValTok{1964}\NormalTok{)}
\KeywordTok{plot}\NormalTok{(output)}
\end{Highlighting}
\end{Shaded}

\includegraphics{timeseries_files/figure-latex/ex2.1-1.pdf}

时序图有递增趋势,非平稳

\begin{Shaded}
\begin{Highlighting}[]
\KeywordTok{acf}\NormalTok{(output, }\DataTypeTok{lag =} \DecValTok{25}\NormalTok{)}
\end{Highlighting}
\end{Shaded}

\includegraphics{timeseries_files/figure-latex/2.1acf-1.pdf}

自相关图衰减缓慢,非平稳。

\begin{center}\rule{0.5\linewidth}{\linethickness}\end{center}

\hypertarget{ux7eafux968fux673aux6027ux68c0ux9a8c}{%
\paragraph{纯随机性检验}\label{ux7eafux968fux673aux6027ux68c0ux9a8c}}

纯随机序列就是白噪声, \[
(1) \quad \mathbb{E}X_t = \mu , \quad \forall\,t\in T
\] \[
(2) \quad \gamma(t,s) = \left \{
    \begin{array}{clc}
      \sigma^2 & , & t = s \\
          0    & , & t \ne s \\
    \end{array}
  \right. ,\quad \forall\,t,s\in T
\]

\begin{center}\rule{0.5\linewidth}{\linethickness}\end{center}

例2.4 随机产生长度为1000的标准正态分布的白噪声序列,并绘制时序图

\begin{Shaded}
\begin{Highlighting}[]
\NormalTok{white_noise <-}\StringTok{ }\KeywordTok{rnorm}\NormalTok{(}\DecValTok{1000}\NormalTok{)}
\NormalTok{white_noise <-}\StringTok{ }\KeywordTok{ts}\NormalTok{(white_noise)}
\KeywordTok{plot}\NormalTok{(white_noise)}
\end{Highlighting}
\end{Shaded}

\includegraphics{timeseries_files/figure-latex/ex2.4-1.pdf}

\begin{center}\rule{0.5\linewidth}{\linethickness}\end{center}

\hypertarget{ux767dux566aux58f0ux5e8fux5217ux7684ux6027ux8d28}{%
\paragraph{白噪声序列的性质}\label{ux767dux566aux58f0ux5e8fux5217ux7684ux6027ux8d28}}

\begin{itemize}
\tightlist
\item
  纯随机性(没有记忆) \[
  \gamma(k) = 0,\quad\forall\,k\ne0
  \]
\item
  方差齐性 \[
  \mathbb{D}X_t = \gamma(0) = \sigma^2,\quad\forall\,k\ne0
  \]
\end{itemize}

\begin{center}\rule{0.5\linewidth}{\linethickness}\end{center}

例2.4续 白噪声序列的\textbf{样本}自相关图

\begin{Shaded}
\begin{Highlighting}[]
\KeywordTok{acf}\NormalTok{(white_noise)}
\end{Highlighting}
\end{Shaded}

\includegraphics{timeseries_files/figure-latex/unnamed-chunk-10-1.pdf}

\begin{center}\rule{0.5\linewidth}{\linethickness}\end{center}

纯随机序列的样本相关系数不会绝对为零,而在0附近随机波动

\hypertarget{barlettux5b9aux7406}{%
\paragraph{Barlett定理}\label{barlettux5b9aux7406}}

纯随机序列,观察期数为\(n\),延迟非零期的样本自相关系数\textbf{近似服从}:
\[
\hat{\rho}_k \, \dot{\sim} \, N(0,\frac{1}{n}),\quad \forall\,k\ne0
\]

**** 纯随机性的检验

对于给定的最大延迟阶数m,\(H_0:\,\rho_1=\rho_2=\cdots=\rho_m=0,\,\,\forall\,m\geqslant1\)

\begin{itemize}
\tightlist
\item
  Q统计量(BP)
\end{itemize}

\[
Q = n\sum\limits_{k=1}^{m}\hat{\rho}^2_k\,\sim\,\chi^2(m)
\]

\begin{itemize}
\tightlist
\item
  LB统计量 \[
  LB = n(n+2)\sum\limits_{k=1}^{m}\left(\frac{\hat{\rho}_k}{n-k}\right)\,\sim\,\chi^2(m)
  \]
\end{itemize}

\begin{center}\rule{0.5\linewidth}{\linethickness}\end{center}

例2.4续 计算白噪声序列延迟6期、延迟12期的\textbf{Q检验}结果

\begin{Shaded}
\begin{Highlighting}[]
\KeywordTok{Box.test}\NormalTok{(white_noise, }\DataTypeTok{lag =} \DecValTok{6}\NormalTok{)}
\end{Highlighting}
\end{Shaded}

\begin{verbatim}
## 
##  Box-Pierce test
## 
## data:  white_noise
## X-squared = 3.4287, df = 6, p-value = 0.7534
\end{verbatim}

\begin{Shaded}
\begin{Highlighting}[]
\KeywordTok{Box.test}\NormalTok{(white_noise, }\DataTypeTok{type =} \StringTok{"Ljung-Box"}\NormalTok{, }\DataTypeTok{lag =} \DecValTok{6}\NormalTok{)}
\end{Highlighting}
\end{Shaded}

\begin{verbatim}
## 
##  Box-Ljung test
## 
## data:  white_noise
## X-squared = 3.4506, df = 6, p-value = 0.7505
\end{verbatim}

\begin{Shaded}
\begin{Highlighting}[]
\KeywordTok{Box.test}\NormalTok{(white_noise, }\DataTypeTok{lag =} \DecValTok{12}\NormalTok{)}
\end{Highlighting}
\end{Shaded}

\begin{verbatim}
## 
##  Box-Pierce test
## 
## data:  white_noise
## X-squared = 15.076, df = 12, p-value = 0.2373
\end{verbatim}

\begin{center}\rule{0.5\linewidth}{\linethickness}\end{center}

例2.5
对1950年------1998年北京市城乡居民定期储蓄所占比例序列的平稳性与纯随机性进行检验

\begin{Shaded}
\begin{Highlighting}[]
\NormalTok{data <-}\StringTok{ }\KeywordTok{read.csv}\NormalTok{(}\StringTok{"timeseries_data/file7.csv"}\NormalTok{, }
                 \DataTypeTok{sep =} \StringTok{","}\NormalTok{, }\DataTypeTok{header =}\NormalTok{ T)}
\NormalTok{prop <-}\StringTok{ }\KeywordTok{ts}\NormalTok{(data}\OperatorTok{$}\NormalTok{prop, }\DataTypeTok{start =} \DecValTok{1950}\NormalTok{)}
\KeywordTok{plot}\NormalTok{(prop)}
\end{Highlighting}
\end{Shaded}

\includegraphics{timeseries_files/figure-latex/ex2.5-1.pdf}

\begin{Shaded}
\begin{Highlighting}[]
\KeywordTok{acf}\NormalTok{(prop)}
\end{Highlighting}
\end{Shaded}

\includegraphics{timeseries_files/figure-latex/ex2.5-2.pdf}

\begin{Shaded}
\begin{Highlighting}[]
\ControlFlowTok{for}\NormalTok{ (i }\ControlFlowTok{in} \DecValTok{1}\OperatorTok{:}\DecValTok{2}\NormalTok{) \{}
  \KeywordTok{print}\NormalTok{(}\KeywordTok{Box.test}\NormalTok{(prop, }\DataTypeTok{lag =} \DecValTok{6}\OperatorTok{*}\NormalTok{i))}
\NormalTok{\}}
\end{Highlighting}
\end{Shaded}

\begin{verbatim}
## 
##  Box-Pierce test
## 
## data:  prop
## X-squared = 68.724, df = 6, p-value = 7.467e-13
## 
## 
##  Box-Pierce test
## 
## data:  prop
## X-squared = 74.74, df = 12, p-value = 4.115e-11
\end{verbatim}

拒绝原假设,该序列不属于白噪声序列

\begin{center}\rule{0.5\linewidth}{\linethickness}\end{center}

\hypertarget{ux7b2cux4e09ux7ae0-ux5e73ux7a33ux65f6ux95f4ux5e8fux5217ux5206ux6790}{%
\section{第三章
平稳时间序列分析}\label{ux7b2cux4e09ux7ae0-ux5e73ux7a33ux65f6ux95f4ux5e8fux5217ux5206ux6790}}

\hypertarget{ux65b9ux6cd5ux6027ux5de5ux5177}{%
\paragraph{方法性工具}\label{ux65b9ux6cd5ux6027ux5de5ux5177}}

\begin{itemize}
\item
  差分运算

  \begin{itemize}
  \item
    一阶差分 \(\nabla x_t = x_t - x_{t-1}\)
  \item
    p阶差分 \(\nabla^p x_t = \nabla^{p-1}x_t - \nabla^{p-1}x_{t-1}\)
  \item
    k步差分 \(\nabla_kx_t = x_t - x_{t-k}\)
  \end{itemize}
\item
  延迟算子 \(B^px_t = x_{t-p}\)
\end{itemize}

\[
(1-B)^n = \sum\limits_{i=0}^{n}(-1)^iC^i_nB^i
\] p阶差分可以表示为 \[
\nabla^p x_t = (1-B)^px_t = \sum\limits_{i=0}^{p}(-1)^iC^i_pB^ix_{t-i}
\]

k步差分 \[
\nabla_kx_t = x_t - x_{t-k} = (1-B^k)x_t
\]

\begin{itemize}
\tightlist
\item
  线性差分方程
\end{itemize}

\hypertarget{armaux6a21ux578b}{%
\paragraph{ARMA模型}\label{armaux6a21ux578b}}

\hypertarget{ux5e73ux7a33ux5e8fux5217ux5efaux6a21}{%
\paragraph{平稳序列建模}\label{ux5e73ux7a33ux5e8fux5217ux5efaux6a21}}

\hypertarget{ux5e8fux5217ux9884ux6d4b}{%
\paragraph{序列预测}\label{ux5e8fux5217ux9884ux6d4b}}


\end{document}
