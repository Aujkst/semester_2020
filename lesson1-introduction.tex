\documentclass[]{article}
\usepackage{lmodern}
\usepackage{amssymb,amsmath}
\usepackage{ifxetex,ifluatex}
\usepackage{fixltx2e} % provides \textsubscript
\ifnum 0\ifxetex 1\fi\ifluatex 1\fi=0 % if pdftex
  \usepackage[T1]{fontenc}
  \usepackage[utf8]{inputenc}
\else % if luatex or xelatex
  \ifxetex
    \usepackage{mathspec}
  \else
    \usepackage{fontspec}
  \fi
  \defaultfontfeatures{Ligatures=TeX,Scale=MatchLowercase}
\fi
% use upquote if available, for straight quotes in verbatim environments
\IfFileExists{upquote.sty}{\usepackage{upquote}}{}
% use microtype if available
\IfFileExists{microtype.sty}{%
\usepackage{microtype}
\UseMicrotypeSet[protrusion]{basicmath} % disable protrusion for tt fonts
}{}
\usepackage[margin=1in]{geometry}
\usepackage{hyperref}
\hypersetup{unicode=true,
            pdftitle={Time Series Analysis},
            pdfauthor={shake\_it},
            pdfborder={0 0 0},
            breaklinks=true}
\urlstyle{same}  % don't use monospace font for urls
\usepackage{color}
\usepackage{fancyvrb}
\newcommand{\VerbBar}{|}
\newcommand{\VERB}{\Verb[commandchars=\\\{\}]}
\DefineVerbatimEnvironment{Highlighting}{Verbatim}{commandchars=\\\{\}}
% Add ',fontsize=\small' for more characters per line
\usepackage{framed}
\definecolor{shadecolor}{RGB}{248,248,248}
\newenvironment{Shaded}{\begin{snugshade}}{\end{snugshade}}
\newcommand{\AlertTok}[1]{\textcolor[rgb]{0.94,0.16,0.16}{#1}}
\newcommand{\AnnotationTok}[1]{\textcolor[rgb]{0.56,0.35,0.01}{\textbf{\textit{#1}}}}
\newcommand{\AttributeTok}[1]{\textcolor[rgb]{0.77,0.63,0.00}{#1}}
\newcommand{\BaseNTok}[1]{\textcolor[rgb]{0.00,0.00,0.81}{#1}}
\newcommand{\BuiltInTok}[1]{#1}
\newcommand{\CharTok}[1]{\textcolor[rgb]{0.31,0.60,0.02}{#1}}
\newcommand{\CommentTok}[1]{\textcolor[rgb]{0.56,0.35,0.01}{\textit{#1}}}
\newcommand{\CommentVarTok}[1]{\textcolor[rgb]{0.56,0.35,0.01}{\textbf{\textit{#1}}}}
\newcommand{\ConstantTok}[1]{\textcolor[rgb]{0.00,0.00,0.00}{#1}}
\newcommand{\ControlFlowTok}[1]{\textcolor[rgb]{0.13,0.29,0.53}{\textbf{#1}}}
\newcommand{\DataTypeTok}[1]{\textcolor[rgb]{0.13,0.29,0.53}{#1}}
\newcommand{\DecValTok}[1]{\textcolor[rgb]{0.00,0.00,0.81}{#1}}
\newcommand{\DocumentationTok}[1]{\textcolor[rgb]{0.56,0.35,0.01}{\textbf{\textit{#1}}}}
\newcommand{\ErrorTok}[1]{\textcolor[rgb]{0.64,0.00,0.00}{\textbf{#1}}}
\newcommand{\ExtensionTok}[1]{#1}
\newcommand{\FloatTok}[1]{\textcolor[rgb]{0.00,0.00,0.81}{#1}}
\newcommand{\FunctionTok}[1]{\textcolor[rgb]{0.00,0.00,0.00}{#1}}
\newcommand{\ImportTok}[1]{#1}
\newcommand{\InformationTok}[1]{\textcolor[rgb]{0.56,0.35,0.01}{\textbf{\textit{#1}}}}
\newcommand{\KeywordTok}[1]{\textcolor[rgb]{0.13,0.29,0.53}{\textbf{#1}}}
\newcommand{\NormalTok}[1]{#1}
\newcommand{\OperatorTok}[1]{\textcolor[rgb]{0.81,0.36,0.00}{\textbf{#1}}}
\newcommand{\OtherTok}[1]{\textcolor[rgb]{0.56,0.35,0.01}{#1}}
\newcommand{\PreprocessorTok}[1]{\textcolor[rgb]{0.56,0.35,0.01}{\textit{#1}}}
\newcommand{\RegionMarkerTok}[1]{#1}
\newcommand{\SpecialCharTok}[1]{\textcolor[rgb]{0.00,0.00,0.00}{#1}}
\newcommand{\SpecialStringTok}[1]{\textcolor[rgb]{0.31,0.60,0.02}{#1}}
\newcommand{\StringTok}[1]{\textcolor[rgb]{0.31,0.60,0.02}{#1}}
\newcommand{\VariableTok}[1]{\textcolor[rgb]{0.00,0.00,0.00}{#1}}
\newcommand{\VerbatimStringTok}[1]{\textcolor[rgb]{0.31,0.60,0.02}{#1}}
\newcommand{\WarningTok}[1]{\textcolor[rgb]{0.56,0.35,0.01}{\textbf{\textit{#1}}}}
\usepackage{graphicx,grffile}
\makeatletter
\def\maxwidth{\ifdim\Gin@nat@width>\linewidth\linewidth\else\Gin@nat@width\fi}
\def\maxheight{\ifdim\Gin@nat@height>\textheight\textheight\else\Gin@nat@height\fi}
\makeatother
% Scale images if necessary, so that they will not overflow the page
% margins by default, and it is still possible to overwrite the defaults
% using explicit options in \includegraphics[width, height, ...]{}
\setkeys{Gin}{width=\maxwidth,height=\maxheight,keepaspectratio}
\IfFileExists{parskip.sty}{%
\usepackage{parskip}
}{% else
\setlength{\parindent}{0pt}
\setlength{\parskip}{6pt plus 2pt minus 1pt}
}
\setlength{\emergencystretch}{3em}  % prevent overfull lines
\providecommand{\tightlist}{%
  \setlength{\itemsep}{0pt}\setlength{\parskip}{0pt}}
\setcounter{secnumdepth}{0}
% Redefines (sub)paragraphs to behave more like sections
\ifx\paragraph\undefined\else
\let\oldparagraph\paragraph
\renewcommand{\paragraph}[1]{\oldparagraph{#1}\mbox{}}
\fi
\ifx\subparagraph\undefined\else
\let\oldsubparagraph\subparagraph
\renewcommand{\subparagraph}[1]{\oldsubparagraph{#1}\mbox{}}
\fi

%%% Use protect on footnotes to avoid problems with footnotes in titles
\let\rmarkdownfootnote\footnote%
\def\footnote{\protect\rmarkdownfootnote}

%%% Change title format to be more compact
\usepackage{titling}

% Create subtitle command for use in maketitle
\providecommand{\subtitle}[1]{
  \posttitle{
    \begin{center}\large#1\end{center}
    }
}

\setlength{\droptitle}{-2em}

  \title{Time Series Analysis}
    \pretitle{\vspace{\droptitle}\centering\huge}
  \posttitle{\par}
    \author{shake\_it}
    \preauthor{\centering\large\emph}
  \postauthor{\par}
    \date{}
    \predate{}\postdate{}
  
%\documentclass{article} 
%\usepackage{ctex} 
\usepackage{latexsym,bm}
\usepackage[BoldFont,SlantFont,CJKchecksingle]{xeCJK}
\setCJKmainfont[BoldFont=SimSun]{Microsoft YaHei} %雅黑
\setCJKmonofont{SimSun}% 设置缺省中文字体
\parindent 2em   %段首缩进

\begin{document}
\maketitle

\hypertarget{ux65b9ux6cd5}{%
\section{方法}\label{ux65b9ux6cd5}}

\begin{itemize}
\tightlist
\item
  \emph{描述性时序分析}
\item
  \emph{统计时序分析}(频域分析方法 + 时域分析方法)
\end{itemize}

\hypertarget{ux751fux6210ux6570ux636e}{%
\section{生成数据}\label{ux751fux6210ux6570ux636e}}

从2005年1月开始的月度数据。\texttt{start}指定起始读入时间,\texttt{frequency}指定序列每年读入的数据频率。

\begin{Shaded}
\begin{Highlighting}[]
\NormalTok{price <-}\StringTok{ }\KeywordTok{c}\NormalTok{(}\DecValTok{101}\NormalTok{, }\DecValTok{82}\NormalTok{, }\DecValTok{66}\NormalTok{, }\DecValTok{35}\NormalTok{, }\DecValTok{31}\NormalTok{, }\DecValTok{7}\NormalTok{)}

\NormalTok{price <-}\StringTok{ }\KeywordTok{ts}\NormalTok{(price, }\DataTypeTok{start =} \KeywordTok{c}\NormalTok{(}\DecValTok{2005}\NormalTok{, }\DecValTok{1}\NormalTok{), }\DataTypeTok{frequency =} \DecValTok{12}\NormalTok{)}
\end{Highlighting}
\end{Shaded}

\hypertarget{ux4f8b1-1}{%
\section{例1-1}\label{ux4f8b1-1}}

读入1884-1939年英格兰和威尔士小麦平均亩产量数据\texttt{file1.csv}

\begin{Shaded}
\begin{Highlighting}[]
\NormalTok{x <-}\StringTok{ }\KeywordTok{read.csv}\NormalTok{(}\StringTok{"D:/Documents/UIBE/6/TimeSeries/file1.csv"}\NormalTok{)}
\KeywordTok{head}\NormalTok{(x)}
\end{Highlighting}
\end{Shaded}

\begin{verbatim}
##   year yield
## 1 1884  15.2
## 2 1885  16.9
## 3 1886  15.3
## 4 1887  14.9
## 5 1888  15.7
## 6 1889  15.1
\end{verbatim}

截取1925年之后的数据\texttt{subset}

\begin{Shaded}
\begin{Highlighting}[]
\NormalTok{z <-}\StringTok{ }\KeywordTok{subset}\NormalTok{(x, year }\OperatorTok{>}\StringTok{ }\DecValTok{1925}\NormalTok{, }\DataTypeTok{select =}\NormalTok{ yield)}
\KeywordTok{head}\NormalTok{(z)}
\end{Highlighting}
\end{Shaded}

\begin{verbatim}
##    yield
## 43  16.0
## 44  16.4
## 45  17.2
## 46  17.8
## 47  14.4
## 48  15.0
\end{verbatim}

对\texttt{yield}序列进行对数变换,并将对数序列和原序列值导出,保存为数据文件\texttt{yield.csv}。

\begin{Shaded}
\begin{Highlighting}[]
\NormalTok{ln_yield <-}\StringTok{ }\KeywordTok{log}\NormalTok{(x}\OperatorTok{$}\NormalTok{yield)}
\NormalTok{x_new <-}\StringTok{ }\KeywordTok{data.frame}\NormalTok{(x,ln_yield) }\CommentTok{#新数据框}
\KeywordTok{write.csv}\NormalTok{(x_new, }\DataTypeTok{file =} \StringTok{"D:/Documents/UIBE/6/TimeSeries/yield.csv"}\NormalTok{, }\DataTypeTok{row.names =}\NormalTok{ F)}
\end{Highlighting}
\end{Shaded}

\hypertarget{ux7f3aux5931ux503cux63d2ux503c}{%
\subsection{缺失值插值}\label{ux7f3aux5931ux503cux63d2ux503c}}

R中缺失值用\texttt{NA}表示。常用的插值方法:\textbf{线性插值}和\textbf{样条插值}

\begin{Shaded}
\begin{Highlighting}[]
\KeywordTok{library}\NormalTok{(zoo)}
\NormalTok{a <-}\StringTok{ }\DecValTok{1}\OperatorTok{:}\DecValTok{7}
\NormalTok{a[}\DecValTok{4}\NormalTok{] <-}\StringTok{ }\OtherTok{NA}
\KeywordTok{cat}\NormalTok{(}\StringTok{"a: "}\NormalTok{, a)}
\end{Highlighting}
\end{Shaded}

\begin{verbatim}
## a:  1 2 3 NA 5 6 7
\end{verbatim}

\begin{Shaded}
\begin{Highlighting}[]
\NormalTok{y1 <-}\StringTok{ }\KeywordTok{na.approx}\NormalTok{(a)}
\NormalTok{y2<-}\KeywordTok{na.spline}\NormalTok{(a)}
\KeywordTok{cat}\NormalTok{(}\StringTok{" "}\NormalTok{, y1, }\StringTok{"}\CharTok{\textbackslash{}n}\StringTok{ "}\NormalTok{, y2)}
\end{Highlighting}
\end{Shaded}

\begin{verbatim}
##   1 2 3 4 5 6 7 
##   1 2 3 4 5 6 7
\end{verbatim}


\end{document}
